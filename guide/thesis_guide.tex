\documentclass[a4paper,twoside,ngerman,UKenglish]{scrbook}
%-------------------------------------------------------------------------------
% This file contains the guide to writing theses for the
% physics and astronomy institutes, Universität Bonn
%
% Specify the language(s) in the class and then use babel.
% If you need more than one language, give the defalt language last,
% e.g. ngerman,UKenglish for a thesis in British (UK) English where you want
% to be able to set the language to German for some part of it

%-------------------------------------------------------------------------------
% Set to <2010 for biblatex and bibtex8 (TeX Live 2009)
% Set to >2010 for biblatex and biber   (TeX Live 2011 or later)
% This must be set before \usepackage{ubonn-thesis}
\newcommand*{\texlive}{2014}
\newcommand*{\guideversion}{3.0}

% Main thesis style file
\usepackage[palatino,bibstyle=alphabetic,feynmp]{../ubonn-thesis}

% Adjustments to biblatex output are in this style file:
\usepackage{../ubonn-biblatex}

\ifthenelse {\texlive < 2010} {%
}{%
  \usepackage{import}
  \usepackage{standalone}
  \usepackage{tikz}
  \usepackage{tikz-3dplot}
  \usepackage{pgfplots}
  \usetikzlibrary{positioning,shapes,arrows}
  \usetikzlibrary{decorations.pathmorphing}
  \usetikzlibrary{decorations.markings}
}
\usepackage{makeidx}
\usepackage[acronym,toc,nosuper]{glossaries}
\usepackage{../thesis_skel/thesis_defs}
\usepackage{guide_defs}

% In order to check if your labels are referenced try the refcheck package
% \usepackage{refcheck}

%-------------------------------------------------------------------------------
% Instead of colouring  links, cites, table of contents etc.
% put them in a coloured box for the screen version.
% This is probably a good idea when you print your thesis.
% \hypersetup{colorlinks=false,
%   linkbordercolor=blue,citebordercolor=magenta,urlbordercolor=darkgreen
% }

%-------------------------------------------------------------------------------
% biblatex is included by ubonn-thesis. Look there for the settings used.
% See the options for settings that can be changed easily.
% For further changes copy the \RequirePackage here and include
% ubonn-thesis with the option biblatex=false.
% Specify the bibliography files here and not at the end!
\addbibresource{guide_refs.bib}
\addbibresource{../guide/refs/zeus_2009.bib}
\addbibresource{../guide/refs/zeus_2010.bib}
% Use example_refs-latin1 and standard_refs-bibtex with bibtex8
% and example_refs-utf8   and standard_refs-biber  if you use biber
\ifthenelse {\texlive = 2009} {%
  \addbibresource{../refs/standard_refs-bibtex.bib}
  \addbibresource{../refs/example_refs-latin1.bib}
}{%
  % Adjustments to output are in this style file:
  \addbibresource{../refs/standard_refs-biber.bib}
  \addbibresource{../refs/example_refs-utf8.bib}
}

%-------------------------------------------------------------------------------
% When writing your thesis it is often helpful to have the date and
% time in the output file. Comment this out for the final version.
%\ifoot[\today{} \thistime]{\today{} \thistime}

%-------------------------------------------------------------------------------
% The following definitions are used to produce the title pages
% needed at various stages
\newcommand{\thesistitle}{Users Guide to Writing a Thesis
  in a Physics/Astronomy Institute
  of the University of Bonn}
\newcommand*{\thesisauthor}{Ian C. Brock}
\newcommand*{\thesistown}{Stoke-on-Trent}
% \renewcommand*{\InstituteName}{\PI}
% \renewcommand*{\inInstitute}{\inPI}
% \renewcommand*{\InstituteAddress}{\PIaddress}
% Adjust \thesisreferee...text depending on male/female referee
\newcommand*{\thesisrefereeonetext}{1.\ Gutachter}
\newcommand*{\thesisrefereeone}{Prof.\ Dr.\ John Smith}
\newcommand*{\thesisrefereetwotext}{2.\ Gutachterin}
\newcommand*{\thesisrefereetwo}{Prof.\ Dr.\ Anne Jones}
% Date when thesis was submitted (Master/Diplom)
% Year or Month, Year when thesis was submitted (PhD)
\newcommand*{\thesissubmit}{XX.YY.2015}
% \newcommand*{\thesissubmit}{Month 2015}
% Date of thesis examination (PhD)
\newcommand*{\thesispromotion}{XX.YY.2015}
% Month and year of the final printed version of the thesis
\newcommand*{\thesismonth}{March}
\newcommand*{\thesisyear}{2015}
\newcommand*{\thesisnumber}{BONN-IR-2015-XXX}

%-------------------------------------------------------------------------------
% The abstract is only needed for the printed version. and should be in
% English regardless of the language of the thesis
\newcommand{\thesisabstract}{%
  \begin{otherlanguage}{UKenglish}
    This document is supposed to provide both a skeleton as well as
    some guidelines as how to format a thesis using \LaTeX\ for the
    Department of Physics and Astronomy of the University of Bonn. The
    recommendations should also be valid for any other department in
    the Faculty of Mathematics and Natural Sciences. The guide
    contains examples and recommendations on what packages to use and
    how to use them. Title pages for PhD, Diplom, Master and Bachelor
    theses are included. It makes no attempt to be an introduction to
    \LaTeX. Indeed some basic \LaTeX{} knowledge is assumed. While the
    guide is primarily geared to (particle) physics theses, it should
    also be useful for others.
  \end{otherlanguage}
}

%-------------------------------------------------------------------------------
% \includeonly can be used to select which chapters you want to process
% A simple \include command just inserts a \clearpage before and after
% the file.
% Note that \includeonly can be quite picky! Do not forget to put a
% comma after the filename, otherwise it will simply be ignored!
% \includeonly{%
%   guide_acknowledge1,
%   guide_intro,
%   guide_tips,
%   guide_submit,
%   guide_package,
%   guide_figs,
%   guide_tables,
%   guide_refs,
%   guide_layout,
%   guide_appendix,
%   guide_acknowledge2
% }

%-------------------------------------------------------------------------------
% Give a list of directories where figures can be found. Do not leave
% any spaces in the list and end the directory name with a /
\graphicspath{%
  {../figs/},%
  {../figs/cover/},%
  {../figs/graphics/}
}

%-------------------------------------------------------------------------------
% Make an index and a glossary
\makeindex
\makeglossaries

% Glossary entries
\input{../guide/guide_glossary}

%-------------------------------------------------------------------------------
\begin{document}
\frontmatter
%-------------------------------------------------------------------------------
% Title page
\input{../cover/Guide_Submit_Title}

\pagestyle{scrplain}

%-------------------------------------------------------------------------------
\include{guide_acknowledge1}

\tableofcontents

\mainmatter
\pagestyle{scrheadings}

%-------------------------------------------------------------------------------
\include{guide_intro}
%\printbibliography[heading=subbibliography]

%==============================================================================
\chapter{Tips and tricks}
\label{sec:tips}\index{tips}
%==============================================================================

\LaTeX{} file: \url{./guide/guide_tips.tex}\\[1ex]
\noindent
Over time I have collected quite a lengthy list of things you should
\enquote{Do} and \enquote{Not Do} (at least in my head) that I think
it is useful to write down early in the document, so that you may
actually read them!  In this chapter I will first tell you how to get
and use the style file and then give some tips.

%------------------------------------------------------------------------------
\section{How to use the \Package{ubonn-thesis} style}
\label{sec:tips:howto}
%------------------------------------------------------------------------------

The idea with this document is that you also look at the \LaTeX{} that
is used to create it, in order to find out how things are done.
I will therefore usually not give the \LaTeX{} commands in the printed
document, but assume that you will have a look at the \LaTeX{} source.
To help you with this, each chapter contains a link to the relevant
file at the top. This link should work if you have compiled the guide
yourself and are in the \texttt{ubonn-thesis} directory.
The files that make up this document are available in a subversion
repository and as a \texttt{tar} file. To get the latest subversion
entries give the command:

\begin{verbatim}
svn co https://svn.physik.uni-bonn.de/basic/ubonn-thesis/trunk
\end{verbatim}
\noindent
If you want to checkout a particular release you can give the command:
\begin{verbatim}
svn co https://svn.physik.uni-bonn.de/basic/ubonn-thesis/tags/ubonn-thesis-N.M
\end{verbatim}
% \noindent
% If you have checked out a version of the guide before 20 August 2012
% and want to update things, you need to change the location:
% \begin{verbatim}
% cd ~/ubonn-thesis/trunk
% svn switch --relocate svn://svn.physik.uni-bonn.de/ubonn-thesis \
%  https://svn.physik.uni-bonn.de/basic/ubonn-thesis
% \end{verbatim}
\noindent
The tar file (and the release tree (as of release 1.5)) also includes
the guide as a PDF file: \texttt{thesis\_guide.pdf}.  It can be
obtained from:\\
\url{http://www-biblio.physik.uni-bonn.de/info/index.shtml#info:titelvorl}
and\\
\url{http://pi.physik.uni-bonn.de/pi_only/thesis.php}.

\par\noindent
Once you have the files, you can then give the command:
\begin{verbatim}
make new [THESIS=dirname]
\end{verbatim}
to create a new directory with a couple of files to help you get
started. By default the directory name will be \texttt{mythesis}.
If you choose a different name, you have to adjust the \Macro{include}
commands inside \texttt{./dirname.tex}.
To compile your thesis try:\index{compiling}
\begin{verbatim}
make thesis [THESIS=dirname]
\end{verbatim}

My original idea was that the style file should work for all recent
\TeX\ installations.  However, some of the packages I recommend have
been changing quite a lot over the past few years. By default things
should work with \TeXLive 2011 and later.\index{TeXLive@\TeXLive!2011}
If you have an older version, you should set the value of
\Macro{texlive} in the thesis main file to
2009\index{TeXLive@\TeXLive!2009} rather than 2011 and then use the
command \enquote{\texttt{make thesis09}} rather than
\enquote{\texttt{make thesis}}. Note that you should not mix these two
commands. If you switch from one to the other do a
\enquote{\texttt{make clean cleanbbl cleanblx}} in between.

\noindent
In the main thesis file, \texttt{mythesis.tex}, you need to specify
which cover page should be used:\index{cover page}
\begin{description}
\item[\texttt{PhD\_Submit\_Title.tex} and \texttt{PhD\_Final\_Title.tex}:]
  the title pages for a PhD thesis;\index{PhD}\index{thesis!PhD}
\item[\texttt{Master\_Submit\_Title.tex} and \texttt{Master\_Final\_Title.tex}:]
  the title pages for an MSc thesis;\index{MSc}\index{thesis!Master}
\item[\texttt{Diplom\_Submit\_Title.tex} and \texttt{Diplom\_Final\_Title.tex}:]
  the title pages for a Diplom thesis;\index{Diplom}\index{thesis!Diplom}
\item[\texttt{Bachelor\_Title.tex}:]
  the title pages for a BSc thesis.\index{BSc}\index{thesis!Bachelor}
\end{description}
For the printed version for the department library you also need to
include the appropriate cover page (with an abstract):\index{cover}
\begin{description}
\item[\texttt{PhD\_Cover.tex}:]
  The cover page for a PhD thesis;
\item[\texttt{Diplom\_Cover.tex}:]
  The cover page for a Diplom thesis;
\item[\texttt{Master\_Cover.tex}:]
  The cover page for an MSc thesis.
\end{description}
This extra cover should not be included for the version of the PhD
thesis that is submitted to the university library (ULB). See
Section~\ref{sec:tips:submit} for some more details.

If you are not a member of the
\foreignquote{ngerman}{Physikalisches Institut} you should
also change \Macro{InstituteName}, \Macro{inInstitute} and
\Macro{InstituteAddress}. The style file \texttt{ubonn-thesis.sty}
already contains the appropriate definitions for PI, HISKP, IAP and
AIFA.

All packages that are needed should be part of your \TeX\
installation. If not you may have to install them or ask your system
administrator to do so.

If you just want to make the cover pages, use the file
\texttt{cover\_only.tex}.  Be sure to adapt the font selected in
\texttt{ubonn-thesis.sty} to the font you actually used in your
thesis. Be aware that not all font sizes are available in all font
collections. If you used the default \LaTeX{} font in your thesis,
then choose \Package{lmodern}\index{font!lmodern} in the style file.

The main file for this guide is \texttt{thesis\_guide.tex} and it
includes the \LaTeX{} files in the directory \texttt{./guide} and the
Feynman graphs in the directory \texttt{./feynmf}. Again this guide
should compile without changes for \TeXLive 2011. For earlier versions
set \Macro{texlive} in \texttt{thesis\_guide.tex} to
2009.\footnote{You should set \Macro{texlive} to 2011 also if you have
  \TeXLive 2012,\index{TeXLive@\TeXLive!2012} which is the default
  version for Ubuntu 12.10.\index{ubuntu!12.10}}


%------------------------------------------------------------------------------
\section{Do}
\label{sec:tips:do}
%------------------------------------------------------------------------------

\begin{itemize}
\item Have 1--2 other people read your thesis well before you are
  supposed to submit it. Do not ask too many; everyone has their own
  opinions on how things should be written, how much detail should be
  included etc.\ and these opinions will not necessarily agree with
  each other!
\item Write \enquote{nice} \LaTeX. It makes it much easier to find mistakes
  in your document. If you like to use the keyboard rather than the
  mouse when moving round in a document, turn on \enquote{auto-fill-mode
  (emacs)}\index{emacs} or its equivalent in any other editor, so
  that line breaks are inserted.
\item Make sure every table and figure is referenced in the text. I
  get very irritated when I suddenly find a figure that is not
  described in the text.
\item Use a units package to format numbers and their units. Recent
  versions of \LaTeX\ include the \PackageG{siunitx} package, which
  is a superior replacement of \Package{SIunits}. I used to
  use \Package{SIunits}, or rather \Package{hepunits} which is
  built on top of \Package{SIunits}, and defines common particle
  physics units such as \si{\GeV}. Use of these packages will be
  discussed in Section~\ref{sec:tips:units}. An alternative is the
  \Package{units} package.
\item Define any complicated symbols once you use them more than
  once:
\begin{verbatim}
\newcommand{\etajet}{\ensuremath{\eta_{\text{jet}}}\xspace}
\end{verbatim}
  If you decide at a later date that jet should be a superscript
  rather than a subscript you only have to change this in one place!
\item If you use normal words in superscripts or subscripts (or
  anything else in math mode) enclose them in \Macro{text}. You can
  also use \Macro{mathrm} or \Macro{textrm}. However, if you then use
  the same symbols in slides with a sans-serif font, the text may well
  continue to be in a
  serif font.\\
  {\sffamily Compare: A common jet energy cut at the LHC is now
    $p_{T}^{\text{jet}} > \SI{20}{\GeV}$, while at HERA we typically
    \ifthenelse {\texlive = 2009} {%
      used $p_{T}^{\mathrm{jet}} > \SI[obeyfamily=false]{6}{\GeV}$.
    }{%
      used $p_{T}^{\mathrm{jet}} > \SI[detect-family=false]{6}{\GeV}$.
    }}
  The \si{\GeV} in the first expression is in sans-serif. This is
  because the
  \Macro{sisetup}\texttt{\{detect-family=true\}}\footnote{\Option{obeyall}
    in \TeXLive 2009.} option is set in \texttt{ubonn-thesis.sty}
  (see Section~\ref{sec:tips:siunitx}).  I used the option
  \Option{detect-family=false}\footnote{\Option{obeyfamily=false} in \TeXLive
    2009} for the \Macro{SI} command in the 2nd expression.
  In the first expression I use \Macro{text} for \enquote{jet}, while
  I use \Macro{mathrm} in the second expression.
\item Use \Macro{xspace} and
  \Macro{ensuremath} in all commands where you would
  like to use a symbol both in text mode and in math mode. Without
  \Macro{xspace} you have to make sure that you end every symbol with
  \texttt{\textbackslash} or \texttt{\{\}}, otherwise the space is used
  to signify the end of the symbol, e.g. in \LaTeX we have to pay
  attention otherwise the symbol and the next word run together.
\item Decide how you want to write abbreviations for particles and
  stick to it -- you should probably define the particle names in your
  style file and always use them (also for quarks). At some point the
  CERN Computer Newsletter claimed that all particle should be written
  upright, e.g. $\text{Z}$ boson, $\text{b}$ quark. However, nowadays
  it seems to be much more common to use math mode. This is also how
  the particles are written in the PDG. I would therefore
  recommend one of the following definitions:
\begin{verbatim}
\newcommand*{\Zo}{\ensuremath{Z}\xspace}
\newcommand*{\Zo}{\ensuremath{\text{Z}}\xspace}
\newcommand*{\bbarQ}{\ensuremath{\bar{b}}\xspace}
\newcommand*{\bbarQ}{\ensuremath{\bar{\text{b}}}\xspace}
\end{verbatim}\index{particles}
  which produce \ensuremath{Z}, \ensuremath{\text{Z}},
  \ensuremath{\bar{b}}, \ensuremath{\bar{\text{b}}}. Note that you are
  not allowed to include numbers in the names of commands, so
  \Macro{Z0} or \Macro{U1S} are not valid commands. For
  $\overline{B}^{\pm}_{c}$ and other particles whose names are capital
  letters, it can be debated whether it better to use
  \Macro{overline} than
  \Macro{bar}. Compare $\overline{B}^{\pm}_{c}$ and
  $\bar{B}^{\pm}_{c}$. If you use upright letters the choice is maybe
  easier: $\overline{\text{B}}^{\pm}_{\text{c}}$ and
  $\bar{\text{B}}^{\pm}_{\text{c}}$ or
  $\overline{\text{K}}^{0}_{\text{S}}$ and
  $\bar{\text{K}}^{0}_{\text{L}}$.

  Note that Kopka~\cite{kopka04}
  recommends using \Macro{newcommand*} rather than \Macro{newcommand}
  for short commands.
\item Decide how you want to write coordinate axes\index{coordinates}
  and stick to it. Far too often I see text like: \enquote{The proton
    beam defines the Z direction, while the interaction point is
    denoted as $(x_{0}, y_{0}, z_{0})$. The polar angle is measured
    with respect to the z axis and $\cot\theta = p_{Z}/\pT$}. Which is
  the best way of writing the coordinates? As $x$, $y$ and $z$ are
  often used for kinematic variables, there are arguments in favour of
  $(X, Y, Z)$.
\item Use \Macro{enquote} from the \PackageG{csquotes} package to
  quote text rather than using explicit quotation marks. This has the
  advantage that consistent quotation marks are used everywhere (also
  if they are nested) and that they are also correct for the language
  (and dialect\footnote{%
    By default, British English\index{British English}\index{UK English}
    uses `British quoted text' for outer quotation marks,
    while American (US) English\index{American English}\index{US English}
    uses \foreignquote{USenglish}{American quoted text}. In this guide
    (and in \texttt{ubonn-thesis.sty})
    \foreignquote{USenglish}{American (US) quotes} are used even if you
    write in British (UK) English.})  you are writing your thesis in. You
  can also see in the Chapter~\ref{sec:intro} how this works, even
  when switching languages inside a paragraph.
\item Use sizes that depend on the font size, \texttt{em}\index{em}
  (width of \enquote{M}) and \texttt{ex}\index{ex} (height of \enquote{x}), for
  spacing that should change with the text size. Use absolute sizes
  \texttt{cm, mm, pt} etc.\ where they are appropriate, e.g.\ title
  pages, boxes, etc. \Macro{quad} and
  \Macro{qquad} are also useful sizes in tables and
  equations.
\item Add a \textbackslash{} (or \{\}) after abbreviations that end with a full
  stop such as e.g.\ if followed directly by text. If you do not, an
  end of sentence space is added rather than a normal interword
  space. Note that \textbackslash{} is not needed (but does no harm)
  after abbreviations that only consist of capital letters:\\
  compare \enquote{e.g. my name is Ian C. Brock}
  and \enquote{e.g.\ my name is Ian C.\ Brock},
  where \textbackslash{} was included in the second version. In this
  example the difference is small. However, if \LaTeX{} increases the
  spacing between words to fill a line the effect is more obvious.
\item Ask someone (me) if you cannot easily find out how to solve
  formatting problems. I recently saw a thesis in German, where the
  author wanted to use commas instead of full stops in numbers and
  wrote numbers as \verb+$2,\!47$+ to produce $2,\!47$ instead of
  $2,47$. There are usually much better solutions, e.g.\
  \verb+\num{2.47}+ with the \Package{siunitx} package produces
  \num{2.47} in English and \foreignlanguage{ngerman}{\num{2.47}} in
  German. In the end such solutions will save you time!
\item Use punctuation in equations and \enquote{\dif}, i.e.\ \Macro{dif} for
  derivatives,\index{derivative} e.g.
  \begin{equation*}
    \int y\, \dif x\,.
  \end{equation*}
  This is also one of the the very few places, where it makes sense
  to put some spacing in by hand -- normally you should leave this to \TeX.
\item Pay attention to the alignment of numbers in
  tables. \Package{siunitx} provides the \Option{S} column specifier to
  help with this. Packages such as \Package{dcolumn} also provide assistance.
\end{itemize}

Not really a strict \enquote{Do}: I highly recommend that you use an
integrated environment for editing and compiling your thesis. If you
are on a Mac or under Windows you probably use TEXshop or
\TeXLive. Under Linux you can use Kile\index{kile} or the way I work
is to use \texttt{emacs}\index{emacs} and AUCTeX. Note that the
RefTeX\index{RefTeX} mode in \texttt{emacs} also provides powerful
tools for finding cross-references and the names of citations
easily. \TeXstudio is based on \TeXmaker and is available for Linux,
Windows and MacOSX. One advantage of such environments is that it is usually
possible to switch between a position output and the relevant place in
the source code and vice versa. This makes it much quicker to fix
things when you spot an error in your output PDF file. You can usually
also step through the errors when you try to compile your file and fix
them directly. In addition, they know which environments and
mathematical symbols exist, which can speed things up if you have not
been working with \LaTeX\ for the past 20 years!

More details on installing \TeX\ for different systems can be found
in Appendix~\ref{sec:app:tex}.


%------------------------------------------------------------------------------
\section{Do Not}
\label{sec:tips:dont}
%------------------------------------------------------------------------------

\begin{itemize}
\item Don't write symbols differently in math mode and in text. One of
  my pet hates is:
  \enquote{The most famous equation in the world is:
  \begin{equation}
    \label{eq:emc2}
    E = m c^{2}
  \end{equation}
  where E is the energy of the particle and m is its mass}, i.e.\ $E$
  and $m$ are in math mode in the equation, but in text mode in the
  text where they are explained.
\item Another example of how not to write things is something like
  \enquote{The scale factor, SF, used to correct the MC is determined in
  an independent dataset using $SF = N_{data} / N_{MC}$}. Note the
  wrong font and spacing of $SF$, $data$ and $MC$. All should be
  enclosed in \Macro{text}: $\text{SF} = N_{\text{data}} / N_{\text{MC}}$.
\item Do not include the directory or the extension of the file in
  \Macro{includegraphics}
  commands. Use
  \Macro{graphicspath}
  instead to set up a list of directories that hold the
  figures.\index{figures!directories} Let \LaTeX{} or PDF\LaTeX{} pick
  the extension for the figure, so that you can (in principle) easily
  switch between the two.
\item Do not try to end paragraphs with
  \textbackslash\textbackslash. These should be used sparingly when
  for some reason you really have to start a new line. Just leave an
  empty line for a new paragraph.
\item Do not draw conclusions or interpret figures in the caption. The
  caption should just describe what is in the figure. Interpretation
  belongs in the main body of the text.
\item Do not start trying to format figure and table captions inside
  each caption -- use the options available in \KOMAScript{} to set
  such things at the beginning of the document.
\item Do not worry about overfull boxes, positions of figures
  and tables, etc.\ until you reach the final version of your thesis.
\end{itemize}


%------------------------------------------------------------------------------
\section{Units}
\label{sec:tips:units}\index{units}
%------------------------------------------------------------------------------

As just mentioned above, but I'll say it again just to make the point,
one of my pet hates is inconsistent and poor typesetting and spacing
of units. At least three standard packages exist to solve this
problem: \Package{siunitx}, \Package{SIunits} and \Package{units}. My
favourite is \Package{siunitx} as it offers many extra features in
addition to the correct typesetting of numbers and their units.

Instead of \Package{SIunits}, I used to use \Package{hepunits}, which
is based on \Package{SIunits} but includes units commonly used in
particle physics such as \si{\GeV} and \si{\pico\barn}. Unfortunately
the syntax of the \Package{hepunits} and \Package{units} packages is
different even though they use the same macro name. In
\Package{hepunits} you write \verb+\unit{10}{\GeV}+, while in
\Package{units} you write \verb+\unit[10]{\GeV}+. The
\Package{siunitx} package is quite new and older versions were
supposed to have compatibility modes for both \Package{SIunits} and
\Package{units}, but I had problems getting them working. It uses the
macros \Macro{SI} and \Macro{num} rather than \Macro{unit}.


%------------------------------------------------------------------------------
\subsection{siunitx package}
\label{sec:tips:siunitx}\index{siunitx}

This package is a more modern and complete package than either
\Package{SIunits} or \Package{units}. As the package is rather new it
is also still developing. There are quite a few changes from version 1
(\TeXLive 2009)\index{TeXLive@\TeXLive!2009} to version 2 (\TeXLive
2011)\index{TeXLive@\TeXLive!2011} which means that several of the
examples I include here have somewhat different syntaxes depending on
which version of \TeXLive is being used to compile this guide. If you
use \TeXLive 2011, but want to use units which were available in
\TeXLive 2009 or the older syntax you can include the option
\Option{version-1-compatibility}. Note that \TeXLive 2009 was the
default version for Ubuntu releases up to and including
12.04,\index{ubuntu!12.04}. I will give the \Package{siunitx} version
2 options in the text and the version 1 options as footnotes.

One very attractive feature is that it allows you to format
computer-generated numbers such as \verb+1.4E4+ automatically,
e.g. \verb+\num{1.4E4}+ produces \num{1.4E4}. Depending on the
language or using the option
\Option{exponent-product}\footnote{\Option{expproduct} in \TeXLive
  2009.} you can also get
\ifthenelse {\texlive = 2009} {%
  \num[expproduct=cdot]{-3.4E-6}.  }{%
  \num[exponent-product=\cdot]{-3.4E-6}.
}
It is even possible to set
the number of decimal places, cf.\ \SI{2.99467E8}{\metre\per\second}
and
\ifthenelse {\texlive = 2009} {%
  \SI[dp=1]{2.99467E8}{\metre\per\second}, }{%
  \SI[round-mode=places,round-precision=1]{2.99467E8}{\metre\per\second},
}
which only differ by the use of the \Option{round-mode} and
\Option{round-precision}\footnote{\Option{dp} in \TeXLive 2009}
options, and even rounds correctly!

Another extremely nice feature of the package is that you can typeset
numbers in a single way and then a full stop or a comma will be used
as the decimal point, depending on which language you set for your
document. This means that computer generated decimal numbers can be
output with commas in a German thesis just by changing the language of
your thesis -- this for me is \LaTeX{} at its best! For example,
writing \verb+\num{1.2345E-3}+ produces \num{1.2345E-3} in the default
language of the document and
\foreignlanguage{ngerman}{\num{1.2345E-3}} if I say that this piece of
text is in German (\Option{ngerman} to be exact).

In keeping with \LaTeX{} philosophy, you can specify a number and its
error using \verb+\num{2.88(32)}+ to produce \num{2.88(32)} with the
\Option{separate-uncertainty}\footnote{\Option{seperr}
  in \TeXLive 2009.} option (which I specify) or $2.88(32)$ with the
default option. You also give as an option how units with negative
powers of units should be shown, e.g.\ per second. This can be changed
for a single command.

Some examples are given below:
\begin{itemize}\setlength{\itemsep}{0pt}\setlength{\parskip}{0pt}
\item $c$ is \SI{3E8}{\metre\per\second} -- default \Macro{per};
\item $c$ is
\ifthenelse {\texlive = 2009} {%
  \SI[per=fraction,fraction=nice]{3E8}{\metre\per\second}
}{%
  \SI[per-mode=fraction,fraction-function=\sfrac]{3E8}{\metre\per\second}
}
  -- using \Option{per-mode=fraction,fraction-function=\Macro{sfrac}}\footnote{%
    \Option{per=fraction,fraction=nice} in \TeXLive 2009}
\item written in \Env{displaymath} or preferably \Env{equation*}:
\ifthenelse {\texlive = 2009} {%
  \begin{equation*}
    c = \SI[per=slash]{2.99E8}{\metre\per\second}
  \end{equation*}
}{%
  \begin{equation*}
    c = \SI[per-mode=symbol]{2.99E8}{\metre\per\second}
  \end{equation*}
}
with \Option{per-mode=symbol}\footnote{%
  \Option{per=slash} in \TeXLive 2009};
\item $\hbar$ is \SI{1.054E-34}{\joule.\second}.
\end{itemize}
You use \enquote{.} or \enquote{,} to make a space between the units,
as illustrated in the last bullet.
Angles are also very straightforward -- just use \Macro{ang} as in
\ang{90}.

If you also want to use \Package{siunitx} in slides, where one usually
uses a sans serif font, you may at first be disappointed that
\Package{siunitx} uses a serif font for the units! DO not despair!
You can use the command
\Macro{sisetup}\texttt{\{detect-family=true\}}\footnote{\Option{obeyall}
  in \TeXLive 2009.} to ensure that the package uses the current font
(in all its aspects) rather than its default.

Have a look at the manual, \texttt{texdoc siunitx}, for many more
examples. \Package{siunitx} also contains useful and powerful tools
for typesetting tables and as mentioned above can be used to round
numbers. These aspects are discussed in Chapter~\ref{sec:table}.

Note that the \Macro{clight} symbol that is used in the macro
\Macro{MeVovercsq} to produce \si{\MeVovercsq} is defined as $c_{0}$ in
\Package{siunitx} version 2. As this is not the way it is usually
written in high energy physics I redefined it using:
\begin{verbatim}
  \DeclareSIUnit\clight{\ensuremath{c}}
\end{verbatim}

Two things that are currently not built in are separate statistical and
systematic errors and asymmetric errors. From the author I got some
suggestions on how to define such things. They are included in
\texttt{guide\_defs.sty} at present and can be added to
\texttt{thesis\_defs.sty}. Two versions of a few of the commands are
defined depending on the value of \Macro{texlive}, as some of the
options changed when moving from version 1 to version 2 of
\Package{siunitx}. Several new macros have been defined there:
\Macro{numerrt}, \Macro{numpmerr} and \Macro{numpmerrt} to write
errors with a description, which are asymmetric and both,
respectively. The corresponding macros for values and errors are
\Macro{SIerrt}, \Macro{SIpmerr} and \Macro{SIpmerrt}. For the macros
whose names end with \enquote{\texttt{t}}, you also have to provide
the descriptive text. If you have two errors then use \Macro{SIerrtt}
or \Macro{SIpmerrtt}, which have two or three more arguments,
respectively. For the standard case of statistical and systematic
errors, you can use the macros \Macro{SIerrs} and
\Macro{SIpmerrs}. Examples of their use are:
\begin{itemize}
\item $\sigma = \SIerrs{3.42}{0.46}{0.32}{\pico\barn}$
\item $\sigma = \SIpmerr{3.42}{0.46}{0.32}{\pico\barn}$
\item $\sigma = \SIpmerrt{3.42}{0.46}{0.32}{\stat}{\pico\barn}$
\item $\sigma = \SIpmerrs{3.42}{0.46}{0.32}{0.06}{0.04}{\pico\barn}$
\item $\sigma = \SIpmerrtt{3.42}{0.46}{0.32}{\stat}{0.06}{0.04}{\sys}{\pico\barn}$
\end{itemize}
The last two examples use \Macro{SIpmerrs} and \Macro{SIpmerrtt} just
to show that they can both give the same output. If you need even more
complicated combinations of errors, or more errors, have a look at the
definitions, e.g.
\begin{equation*}
  \sigma_{t\bar{t}} = (\num{164.6}%
  \valuesep\numerrt{8.7}{\stat}%
  \valuesep\numpmerrt{6.4}{5.3}{\sys}%
  \valuesep\numerrt{8.2}{lumi.})%
  \valuesep\si{\pico\barn}
\end{equation*}


%------------------------------------------------------------------------------
\subsection{SIunits/hepunits packages}
\label{sec:tips:siunits}\index{SIunits}\index{hepunits}

Before I found \Package{siunitx} these used to be my preferred units
packages.
% While \Package{siunitx} is supposed to have a compatibility
% mode for \Package{hepunits} I had problems getting this working.
This section therefore gives examples on how to use \Package{hepunits}
and what you should be careful about. As \Package{siunitx} has got
stricter about what it allows for a syntax, I have had to cheat in the
\LaTeX\ code several times to show the effect.

Even though \Macro{xspace} is used for some units in \Package{hepunits}
it does not appear to have the usual effect.
Hence, if you use units in normal text it is probably wise to
terminate them with \enquote{\textbackslash } or \enquote{\{\}}. Compare
\begin{itemize}
\item The \si{\GeV}is a
  heavily used unit in particle physics and cross-sections measured in
  \si{\pico\barn}or \si{\nano\barn}are quite common.
  Masses can be given in either \si{\MeVovercsq}or \si{\MeV}.
\item The \si{\GeV} is a
  heavily used unit in particle physics and cross-sections measured in
  \si{\pico\barn} or \si{\nano\barn} are quite common.
  Masses can be given in either \si{\MeVovercsq} or \si{\MeV}.
\end{itemize}
In the first bullet the units were not terminated, while in the second
they were.

Note that \Package{SIunits} typesets the value in text mode and the
unit in math mode. You only have to worry about this if you want to
use math mode symbols e.g. $10^{8}$ in the value. Three different ways
of writing the velocity of light are:
\begin{itemize}\setlength{\itemsep}{0pt}\setlength{\parskip}{0pt}
\item $c$ is \verb+\unit{$3 \cdot 10^{8}$}{\metre\per\second}+
\item $c$ is
  \verb+\unit{3$\cdot$\power{10}{8}}{\metre\reciprocal\second}+
\item $c$ is \verb+\unit{3 $\cdot$ \power{10}{8}}{\metre\usk\reciprocal\second}+
\item Written in \Env{displaymath} or preferably \Env{equation*}:
\begin{verbatim}
  \begin{equation*}
    c = \unit{$3 \times \power{10}{8}$}{\metre\usk\reciprocal\second}
  \end{equation*}
\end{verbatim}
\end{itemize}
Note the difference in spacing in the examples. The first
example gives the best result. Two different ways of handling the value
are used in the second and third examples, either without or with space between
the number and \verb+$\cdot$+. In the fourth example, where you use
\Macro{unit} in math mode, you still need to enclose the value in
\$...\$ if it includes characters from math mode.  Conversely if for
some reason you want normal text in your units, you should put it
inside \Macro{text}, e.g. the velocity of light is
$\SI{3E8}{\text{metres per second}}$. If you forgot the
\Macro{text} command using\\
\verb+\unit{$3 \cdot \power{10}{8}$}{metres per second}+ you would
get:
$3 \cdot 10^{8}\,metres per second$ or if you tried to write the units
yourself you would get $3 \times 10^8\,m s^{-1}$.

If you have negative powers then you can play around with
the \Macro{power} command and the usual superscript:
\begin{itemize}
\item $\hbar$ is \verb+\unit{$1.054 \times 10^{-34}$}{\joule\usk\second}+
\item $\hbar$ is \verb+\unit{1.054 $\times$ \power{10}{-34}}{\joule\usk\second}+
\item $\hbar$ is \verb+\unit{1.054 $\times$ \power{10}{$-34$}}{\joule\usk\second}+
\end{itemize}
Note the use of \Macro{usk} to get a bit of space between the
units. In the second example the minus sign appears as a dash,
\enquote{-}, rather than \enquote{$-$}, which is too small. Hence, if you
use \Macro{power} you should put the power in math mode.

Just to complicate things further, if you use the Palatino font for
example, then the standard \TeX{} font is used in math mode. You thus
have to decide from the very beginning whether numbers should
\textbf{ALL} be in text or in math mode. This is clearly one of the
disadvantages of using a font for which the math mode is different
from the text mode. ATLAS uses either the \Package{txfonts} or the
\Package{mathptmx} packages, which do not have this problem. This is
why I use \Package{txfonts} in this guide. If the numbers
$1234.56$ and 1234.56 look the same then you do not have to
worry! There are ways to get around the problem with Palatino. You can
try either the \Package{pxfonts} or the \Package{mathpazo} packages --
for my taste the sans serif font used in \Package{pxfonts} looks a bit better.


%------------------------------------------------------------------------------
\section{Hints}
\label{sec:tips:hints}
%------------------------------------------------------------------------------

\Package{xspace} is great, but how do you write $B^{0}\bar{B}^{0}$ when
you have defined the symbols \Macro{Bo} and \Macro{Bobar} with
\Macro{xspace} at the end. Here you need \verb+{}+ between the two
commands, e.g. \verb+\Bo{}\Bobar+ produces \Bo{}\Bobar while
\verb+\Bo\Bobar+ produces \Bo\Bobar.

How should you write \enquote{between $10^{4}$ and $10^{5}$}?\index{range}
If you use math mode it looks like $10^{4} - 10^{5}$, which is not
really what you want. Although it is rather clumsy, the best way to do
it is \verb+$10^{4}$--$10^{5}$+ which produces
\enquote{$10^{4}$--$10^{5}$}.

With \Package{siunitx} and \Macro{num} there
is a built-in option. You simply write\\
\verb+\numrange{3e4}{7e4}+
to produce \enquote{\numrange{3e4}{7e4}} or\\
\ifthenelse {\texlive = 2009} {%
  \texttt{\textbackslash SIrange[tophrase=-\,-]{5}{7}{\textbackslash GeV}}
  to produce \enquote{\SIrange[tophrase=--]{5}{7}{\GeV}}.
  If you only want the unit to appear once write\\
  \texttt{\textbackslash SIrange[range-units = single, tophrase=-\,-]\{5\}\{7\}\{\textbackslash GeV\}}
  to produce
  \SIrange[trapambigrange=false, tophrase=--]{5}{7}{\GeV}.
}{%
  \texttt{\textbackslash SIrange[range-phrase=-\,-]\{5\}\{7\}\{\textbackslash GeV\}}
  to produce \enquote{\SIrange[range-phrase=--]{5}{7}{\GeV}}.
  If you only want the unit to appear once write\\
  \texttt{\textbackslash SIrange[range-units = single, range-phrase=-\,-]\{5\}\{7\}\{\textbackslash GeV\}}
  to produce
  \SIrange[range-units = single, range-phrase=--]{5}{7}{\GeV}.
}
In a single range, such settings are rather long-winded!
However, they can be applied to the whole
document using the \Macro{sisetup} macro.

With \Package{SIunits} and \Macro{unit} you can write
\verb+\unit{\power{10}{4}--\power{10}{5}}{}+, which would look almost
correct, but leaves some space for the unit! If you want to write
between 5 and \SI{7}{\GeV}, then \Macro{unit} works well:
\verb+\unit{5--7}{\GeV}+.

A similar problem is how do you write \enquote{about 10\%}?  Again the
simple solution $\sim 10\%$ or $\sim$ 10\% have too much space. In the
file \texttt{thesis\_defs.sty} two macros \Macro{mysim} and
\Macro{mysymeq} are defined that add some negative space so that you
can simply put everything in math mode: $\mysim 10\%$ or $\mysimeq
0.2$. An alternative is to use \Macro{SI} or \Macro{unit}, as there
should actually be some space between the number and the \% sign,
e.g.\ \verb+$\sim$\SI{10}{\%}+, which produces $\sim$\SI{10}{\%}, is
completely correct and does not need the use of \Macro{mysim}.

What is the difference between \Macro{textrm} and
\Macro{mathrm}? I used to worry about this and found a
few examples (which I then forgot) where the font size was better
using one or the other. Then I learnt about \Macro{text},
converted all my predefined symbols to use \Macro{text} rather than
\Macro{mathrm} or \Macro{textrm}, and don't have to worry any
more. However, I can given an example: the transverse energy of the
highest energy jet is denoted $p_{T}^{\mathrm{1^{\text{st}} jet}}$ if
one uses \Macro{mathrm}, while it is denoted by
$p_{T}^{\textrm{1$^{\text{st}}$ jet}}$ if you use \Macro{textrm},
where I used \Macro{text} to produce $1^{\text{st}}$.  As you can see
from this example, the key difference is that \Macro{mathrm} switches
to an upright font, but keeps you in math mode -- hence ignoring any
spaces. \Macro{textrm} switches to text mode (with a serif font) and
therefore pays attention to spaces.

A perennial problem is bold math\index{math!bold} when it is needed in
section headings etc. This is further complicated by the fact that the
table of contents is usually not typeset using a bold font (except for
chapter titles, or whatever the highest level(s) of sectioning are.
The only reliable way to get this right in both cases is to give the
heading twice, once with \Macro{boldmath} for the real title and once
without as the optional title for the table of contents.  An
illustration of how this works is given in Appendix~\ref{sec:emc2}.



%------------------------------------------------------------------------------
\section{Line numbering}
\label{sec:tips:lineno}\index{lineno}\index{line numbers}
%------------------------------------------------------------------------------

You probably do not need line numbers when writing your
thesis (who knows?). However, as this is a very useful package that
occasionally has some problems, I thought I would include some
information here.

\Package{lineno} is the package to use to get line numbers in your text,
but sometimes a block of lines is not numbered - see Fig.~\ref{fig:lineno}a.

\begin{figure}[htbp]
  \centering
  \begin{tabular}{cc}
  \includegraphics[width=7cm]{BCDijet} &
  \includegraphics[width=7cm]{BCDijet-linenomath}\\
  (a) & (b)
  \end{tabular}
  \caption{Example of (a) a problem with line numbers and (b) its solution.}
  \label{fig:lineno}
\end{figure}

Such problems are associated with text that is close to math mode
environments. Some of the problems can be solved by using a new
version of the lineno package.
However, this only works for \enquote{standard} \LaTeX{}
math environments: \Env{displaymath}, \Env{equation} and \Env{eqnarray}, while it does
not work for recommended \Package{amsmath} environments such as \Env{equation*},
\Env{align(*)} and \Env{alignat(*)}.

The solution is to enclose the equation in \Env{linenomath} environment, e.g.
\begin{verbatim}
The total visible cross section for inclusive heavy-quark jet
production, $\sigma^{q}$, with $q\in\{b,c\}$ is given by
\begin{linenomath}
\begin{equation*}
  \sigma^{q} = \frac{N_{q}^{\text{rec,Data}}}{\mathcal{A}_{q}\cdot\mathcal{L}_{\text{Data}}}.
\end{equation*}
end{linenomath}
Here, $\mathcal{L}_{\text{Data}}$ denotes the integrated luminosity,
$\mathcal{A}_{q}$ is the acceptance and $N_{q}^{\text{rec,Data}}$ the
number of reconstructed heavy-quark jets in data, which was determined
\end{verbatim}

Then the line numbering will be correct, see Fig.~\ref{fig:lineno}b.

%%% Local Variables:
%%% mode: latex
%%% TeX-master: "../thesis_guide"
%%% End:

%\printbibliography[heading=subbibliography]

\include{guide_submit}
%\printbibliography[heading=subbibliography]

%==============================================================================
\chapter{Useful packages}
\label{sec:package}
%==============================================================================

\LaTeX{} file: \url{./guide_package.tex}\\[1ex]
\noindent
\LaTeX{} has so many packages that it is often hard to find the
correct or most useful ones. It is also not a good idea to just take
one of your friend's theses and use his/her packages and conventions,
as there is a steady and regular improvement in the packages
available.

This chapter lists some useful packages -- maybe also some that
are not so commonly known. Here I only say what the package is used
for. More detailed instructions on the usage can be found in the
relevant chapters. I first list the packages used in this guide
and then give a bit of information on other packages that may be useful.

From all that I have read, \KOMAScript\index{KOMAScript@\KOMAScript}
seems to be the way to go for the overall classes. I have therefore
based the \Package{ubonn-thesis} style on this. You replace
\Package{article}, \Package{report} and \Package{book} by
\Package{scrartcl}, \Package{scrreprt} and \Package{scrbook}. For
theses I think it is best to use \Package{scrbook}, as this class also
includes the commands \Macro{frontmatter}, \Macro{mainmatter} and
\Macro{backmatter} that set up page numbering etc.\ appropriately.

Please try to use \KOMAScript\ version 3.0 or higher. The
\Macro{KOMAoptions} command is not available in earlier versions, so
you would have to modify the style file.


%------------------------------------------------------------------------------
\section{Layout and language}
\label{sec:package:layout}
%------------------------------------------------------------------------------

There are quite a few packages related to layout and also to handling
of text input and languages. As far as layout goes, \KOMAScript{} has
many options with which you can already do a lot. You can either use
the built-in \Package{typearea} package to do the page layout, which
also includes nice options to allow for the binding, or use the
\Package{geometry} package which also contains more than enough
options. In the past I have used \Package{geometry}, but I also see no
reason not to just use \Package{typearea}. Note that you should not
include the \Package{typearea} package, you should simply set the
options using \Macro{KOMAoptions}.
The packages are listed in Table~\ref{tab:package:layout}.

\begin{table}[htbp]
  \centering
  \begin{tabular}{lp{0.8\textwidth}}
    \toprule
    \Package{geometry} & Provides simple options for page layout such
    as \Option{scale=0.75} to cover 75\% of the page.\\
    \Package{typearea} & Does much the same, but here you specify how
    many elements to split the page into. You do not have to include
    this package explicitly if you use \KOMAScript.\\
    \Package{setspace} & Useful options to change spacing.\\
    \Package{fontenc} & The encoding used for fonts. Recommended is
    \Option{T1}, which is given as an option.\\
    \Package{inputenc} & Use either \Option{utf8} or \Option{latin1} so
    that you can input German letters such as ä, ü and ß directly.\\
    \Package{babel} & Language specific typesetting.\\
    \Package{csquotes} & Package for quoting things using the correct
    language-dependent symbol.\\
    \Package{scrpage2} & Set headers and footer.\\
    \Package{xspace} & Avoid having to put
    \enquote{\texttt{\textbackslash\ }} or
    \enquote{\texttt{\{\}}} after a macro.\\
    \bottomrule
  \end{tabular}
  \caption{Useful packages for layout.}
  \label{tab:package:layout}
\end{table}

%------------------------------------------------------------------------------
\section{Appearance}
\label{sec:package:appearance}
%------------------------------------------------------------------------------

It used to be the case that nearly all \LaTeX{} documents used the
Computer Modern Fonts. That is no longer necessary. There are rather
complete font sets that are also free that you can use instead.
%In this guide I use a combination of \Package{Palatino} and other fonts.
The default font for theses is \Package{txfonts}.
\Package{newtx} is a new version of this package that is available
as of \TeXLive 2013.
If you have this package, then I would recommend using it.
Some of the spacings in equations have been improved 
and there is a better balance of the sizes of serif, \textsf{sans serif} and \texttt{typewriter} fonts.
Other fonts that look quite nice (e.g.\ Palatino) can also be used.
The option \Option{palatino} in \Package{ubonn-thesis} can be used to select this.
The option actually selects the font packages \Package{mathpazo}, \Package{courier} and \Package{helvet}.
Another alternative is a package such as \Package{pxfonts} 
to get both text and math fonts in the same style. 
Some examples of other possible font packages are given in the style file.
As mentioned above, certain fonts can be selected directly via options:
\Option{txfonts}, \Option{newtx} or \Option{palatino}.

Commonly used packages associated with fonts, tables and
figures are listed in Table~\ref{tab:package:appearance}.

\begin{table}[htbp]
  \centering
  \begin{tabular}{lp{0.8\textwidth}}
    \toprule
    \Package{siunitx} & Typeset units properly with correct spacing.\\
    \Package{graphicx} & The package to use for including graphics.\\
    \Package{rotating} & Package to use for rotating tables etc. The
    \Macro{includegraphics} command can rotate figures directly.\\
    \Package{array} & Adds extra column formatting capabilities.\\
    \Package{xtab} & Can produce tables that extend over more
    than one page.\\
    \Package{booktabs} & Help for producing nicer tables.\\
    \Package{amsmath}, \Package{amssymb} & Extra math commands and symbols from
    AMS.\\
    \Package{mhchem} & Nice package for typesetting chemical elements
    correctly.\\
    \Package{xfrac} & Some more options for typesetting fractions.\\
    \Package{xcolor} & Add colour commands.\\
    \Package{titlesec} & Change the appearance of chapter and section
    headings. See below for more information.\\
    \bottomrule
  \end{tabular}
  \caption{Useful packages for appearance.}
  \label{tab:package:appearance}
\end{table}

As alternatives to \Package{xtab}, one can also use
\Package{supertabular} or \Package{longtable}. 
All these packages also have the advantage that you
can specify header and footer text.
If you use the \Env{mpxtabuar} environment from \Package{xtab} 
you can include footnotes in a table.
See the \Package{xtab} documentation for more details.
It is probably best to only use one of these three packages to avoid conflicts.

Use of the \Package{titlesec} package gives a warning when using \KOMAScript{}; hence
as of version 3.0 the chapter title formatting is done by hand in \Package{ubonn-thesis}.
You can switch back to using \Package{titlesec} by giving the option \Option{titlesec}.


%------------------------------------------------------------------------------
\section{Other packages}
\label{sec:package:other}
%------------------------------------------------------------------------------

Some other useful packages, some of which are included in
\texttt{ubonn-thesis.sty} are listed in
Table~\ref{tab:package:other1}.

\begin{table}[htbp]
  \centering
  \begin{tabular}{lp{0.8\textwidth}}
    \toprule
    \Package{ifthen} & Provides the \Macro{ifthenelse} command.\\
    \Package{IEEEtrantools} & Contains useful environments for multiline equations.\\
    \Package{feynmp} & Draw Feynman graphs with Metapost.\\
    \Package{axodraw} & Draw Feynman graphs.\\
    \Package{tikz} & General drawing package that can also be used for
    Feynman graphs.\\
    \Package{standalone} & Allows you to have a document that you can
    directly compile for each figure and also input to another document.\\
    \Package{hyperref} & Adds \Macro{url} command as well as ability
    to click on entries on table of contents etc.\\
    \Package{subfiles} & Provides a nice alternative to
    \Macro{include}.\\
    \Package{subcaption} & An newer alternative to \Package{subfig}.\\
    \Package{tabularx} & Allows fixed table width with flexible column
    widths.\\
    \Package{floatrow} & Add ability to define own floats.\\
    \Package{commath} & Some useful extra maths commands, especially for differentials.\\
    \Package{skmath} & More maths commands that could be very useful.\\
    \Package{adjustbox} & Add much more sophisticated clipping
    capabilities than offered by \Package{graphicx}.\\
    \Package{wrapfig} & Allow text to flow around figures.\\
    \Package{floatflt} & Similar capabilities to \Package{wrapfig} -- allow text to flow around figures.\\
    \Package{glossaries} & Provide commands for creating a
    glossary. This is intended to replace the \Package{glossary}
    package.\\
    \Package{dcolumn} & Very helpful for lining up columns on character
    strings such as a decimal point. \Package{siunitx} offers similar
    and better functionality.\\
    \Package{refcheck} & Check whether labels are used, i.e.\ if figures
    and tables are actually referenced.\\
    \bottomrule
  \end{tabular}
  \caption{Other useful packages.}
  \label{tab:package:other1}
\end{table}

While the \Package{amsmath} package solves many problems that occur if
you just use the normal \LaTeX{} math mode commands, there are some
things that are not so nice with long and complicated multiline
equations. \Package{IEEEtrantools}, in particular the
\Env{IEEEeqnarray} environment, can help here. See Ref.~\cite{lshort}
or the package documentation for more details.

\Package{standalone} is both a package and a document class. It is
available from \TeXLive 2012 onward. It allows you to have a
standalone document for a \Package{tikz} or \Package{feynmf} figure
and also input this file into another document. If you run PDF\LaTeX\
on the file it also automatically crops the resulting picture. This is
one of those packages where you think \enquote{Why didn't someone
  create this years ago?}. The \Package{tikz} figures included in this
guide make use of it.

I do not recommend the \Package{subfiles} package by default as it is not
included in \TeXLive by default. Have a look at
\url{http://en.wikibooks.org/wiki/LaTeX/General_Guidelines} for
example for more information. If you want to use the package you have
to download it and install for yourself. You can do much the same
thing using AUCTeX inside \texttt{emacs}.

The packages \Package{commath} and \Package{skmath} provide some additional maths command.
Differentials etc.\ are particularly useful. 
\Package{skmath} does quite a lot more and even modifies/enhances some standard commands.

A list of other packages that are commonly used is given in
Table~\ref{tab:package:other2}. They are not
included in the list above, because they are either not really needed
or have been superseded by other packages.

\begin{table}[htbp]
  \centering
  \begin{tabular}{lp{0.8\textwidth}}
    \toprule
    \Package{hepunits} & Typeset units properly with correct spacing\\
    \Package{units} & Typeset units properly with correct spacing.\\
    \Package{SIunits} & Typeset units properly with correct
    spacing. \Package{hepunits} uses this package, so it does not need
    to be added explicitly\\
    \Package{fancyhdr} & As the name suggests, do your own header and
    footer configuration. Within \KOMAScript{} it is recommended to use
    \Package{scrpage2} instead.\\
    \Package{feynmf} & Draw Feynman graphs with Metafont\\
    \Package{subfig} & As the name suggests make sub-figures and add
    separate captions for them. \emph{This package has apparently been
      deprecated.}\\
    \Package{subfigure} & As the name suggests make sub-figures and add
    separate captions for them. \emph{This package is deprecated.}\\
    \Package{color} & Add colour commands -- \Package{xcolor} is
    needed to colour boxes around links.\\
    \Package{float} & As far as I can tell \Package{floatflt} offers more options.\\
    \Package{caption} & Much more control on captions -- as
    \KOMAScript{} also has many options, not sure this is necessary.\\
    \Package{ziffer} & Spacing with a comma as decimal separator is
    correct.\\
    \Package{nomencl} & Another package for creating a glossary.\\
    \Package{fncychap} & Another package for changing the style of the
    chapter heading.\\
    \Package{quotchap} & Another package for changing the style of the
    chapter heading.\\
    \bottomrule
  \end{tabular}
  \caption{Other packages that are often used, but I have already
    given alternatives.}
  \label{tab:package:other2}
\end{table}

As indicated, the \Package{ziffer} package is advertised as providing
the correct spacing after a comma in math mode if you use the comma as
the decimal separator. Compare $2,5$ with 2,5 and $2{,}5$. The first
spacing is wrong. If you use the \Package{ziffer} package it will be
correct. However, it does seem to conflict with the use of the
\Package{dcolumn} package, so I cannot compile this guide
if \Package{ziffer} is included. Some  workarounds are discussed in
Section~\ref{sec:layout:german}. In addition, the \Package{siunitx}
package contains the same functionality, which can simply be steered
by changing the document language, as discussed in
Section~\ref{sec:table:siunitx}. Hence, \Package{ziffer} is not really
needed anymore.


%%% Local Variables:
%%% mode: latex
%%% TeX-master: "./thesis_guide"
%%% End:

%\printbibliography[heading=subbibliography]

\include{guide_figs}
%\printbibliography[heading=subbibliography]

\include{guide_tables}
%\printbibliography[heading=subbibliography]

%==============================================================================
\chapter{References}
\label{sec:ref}
%==============================================================================

\LaTeX{} file: \url{./guide/guide_refs.tex}\\[1ex]
\noindent
Every thesis should also include a list of references, called the
bibliography in \LaTeX{} terminology. You \enquote{cite} a reference using
the \Macro{cite} command. For example, the book of
Kopka~\cite{kopka04} is my favourite \LaTeX{} book. In general you
should include a non-breaking space, i.e.\ \enquote{\textasciitilde} between
the text and the \Macro{cite} command. In British (UK) English the reference
should come before the punctuation; in American English it comes after
it.\footnote{After some research, it appears to me that the footnote
  number should come after the punctuation, unless the footnote only
  refers to the last word of the phrase or sentence.}

That is the easy part!  Where do you get the references from and how
do you format them? Sources of references are discussed in
Section~\ref{sec:ref:sources}.
There are two options for the formatting. Either you do it by
hand, formatting \Macro{bibitem} entries yourself or you use
\BibTeX. While \BibTeX\ may appear to be the more complicated option
at the beginning, I strongly recommend that you use it.

In addition, you have to make sure that authors' names are
printed consistently, you include the appropriate collaboration
name, the title is formatted correctly and journals are given
consistent abbreviations. Such topics are discussed in
Section~\ref{sec:ref:bib}.

What about citing a series of articles? Can you
include them in one reference or do you want to keep one article per
reference?
I give some hints on
useful options and settings for \Package{biblatex} below (Section~\ref{sec:ref:bbx}).
If you use \BibTeX, then you probably have to use the \Package{mcite}
package -- see Section~\ref{sec:ref:mcite}.
Just to take a silly example. The ZEUS collaboration
publications in
2010~\cite{Abramowicz:2010ih,Abramowicz:2010xc,Abramowicz:2010nj} were
not as numerous as in previous years. If you use the standard
\Macro{cite} command and the \Option{unsrt} option or its equivalent,
you get a list of numbers.  In the past one could use the
\Package{mcite} package to make the references nicer, put them all in
one, write the list as [m--n] etc.


%------------------------------------------------------------------------------
\section{Formatting by hand}
\label{sec:ref:bibitem}
%------------------------------------------------------------------------------

Don't! The number of references that you will need will probably grow
fast. It is quite likely that at some point you will decide that they
are not really formatted as you would like them to be. You will almost
certainly add references when you correct your thesis. How do you
make sure they are in the order you want? How do you make sure that
only articles that you actually refer to are in the bibliography?

Suppose you want to use some of the references in your thesis in
conference proceedings or a paper in a journal. Every place where you
publish will have it's own preferred format for the references that
almost certainly will not be the one you chose for your thesis.

If you insist on following this route, consult a book on \LaTeX!


%------------------------------------------------------------------------------
\section{Using \BibTeX\ and \Package{biblatex}}
\label{sec:ref:bibtex}\index{BibTeX@\BibTeX}
%------------------------------------------------------------------------------

I won't pretend that \BibTeX{} is the most user-friendly way of
handling references and there are several things that you have to pay
attention to when you use it for your references.

The two big advantages of \BibTeX{} are: only references that you
actually refer to appear in the bibliography; you can change the
format (consistently) of all articles in the bibliography simply by
changing the style!

The first step is to put your references in one or more \texttt{.bib}
files. For this document they can be found in:
\begin{itemize}
\setlength{\itemsep}{0pt}
\item \url{./guide/guide_refs.bib};
\item \url{./refs/standard_refs-bibtex.bib} or
  \url{./refs/standard_refs-biber.bib};
\item \url{./refs/example_refs-latin1.bib} or
  \url{./refs/example_refs-utf8.bib};
\item \url{./guide/refs/zeus_2009.bib} and
  \url{./guide/refs/zeus_2010.bib}.
\end{itemize}
For each article you specify things like its
title, author, journal etc.

You then include these files into your \LaTeX{} document where you
want the bibliography to appear and specify which style should be
used.

At this point you also have to decide which interface to the contents
of the \texttt{.bib} files you want to use. You have a choice of the
original \BibTeX\index{BibTeX@\BibTeX} or the more modern
\Package{biblatex}. If you use \Package{biblatex} you
need something like:
\begin{verbatim}
%
% Use biblatex for the bibliography
%
\usepackage[backend=bibtex8,hyperref=true,bibencoding=latin1,
  style=numeric-comp,sorting=none,block=ragged,firstinits=true]{biblatex}
% \usepackage[backend=biber,
%   style=numeric-comp,sorting=none,block=ragged,firstinits=true]{biblatex}
%
% Adjustments to output are in this style file:
\usepackage{./biblatex/biblatex-num-v2009}
% \usepackage{./biblatex/biblatex-num-v2011}
\bibliography{./mythesis/thesis_refs.bib,%
  ./refs/standard_refs-bibtex.bib}
% \addbibresource{./mythesis/thesis_refs.bib}
% \addbibresource{./refs/standard_refs-biber.bib}
\end{verbatim}
in the document preamble and \Macro{printbibliography} where they
should be printed. The commented out lines are for \TeXLive 2011 with
the \Option{biber} backend.
\par\noindent
If you use \BibTeX{} you need something like:
\begin{verbatim}
%
% Use BibTeX for the bibliography
%
\bibliographystyle{unsrt}
\bibliography{./mythesis/thesis_refs,%
  ./refs/standard_refs-bibtex}
\end{verbatim}
at the point where the references should be printed.
Note that \LaTeX{} is sometimes picky about lists of directories that
have spaces between them, so it is safer to include all files on one line.

Which should you use? \BibTeX{} has been around for a long time and is
therefore better known. However, it has many problems when it comes to
sorting, handling more modern sources of information (e.g.\ the web),
etc. \Package{biblatex} is still developing rather quickly and
so some options and ways of doing things may change. It supports
things like online references and enables you to click on references
using the preprint number or DOI to look at a reference. It is also
easier though to change the way your references look. I therefore
strongly recommend that you use \Package{biblatex}.

Another serious problem with \BibTeX{} is that it cannot handle
umlauts etc.\ properly. While I have said elsewhere in this document
that you should use UTF-8 or latin1 encoding so that you can enter ä
etc.\ directly, this does not work with \BibTeX. You can use the
old syntax \verb+\"{a}+. This problem is completely solved if you use
\Package{biblatex} and the \Option{biber} backend.

If you use \Package{biblatex} and the \Option{bibtex8} backend (default
setting for \TeXLive 2009), then you have to encode your \texttt{.bib}
files with latin1 and use the line:
\Macro*{usepackage[bibencoding=latin1]\{biblatex\}}.\footnote{%
  If you do not include umlauts directly in the references you do not
  need this option.}
For this reason, as indicated above the thesis package includes two
versions of the file with some standard references and also of the
files with some example references:
\begin{itemize}
\setlength{\parskip}{0pt}\setlength{\itemsep}{0pt}
\item \texttt{./ref/examples\_refs-latin1.bib} with latin1 encoding
\item \texttt{./ref/examples\_refs-utf8.bib} with UTF-8 encoding.
\end{itemize}
If you try to compile the guide with the wrong file using \BibTeX, you
will get some errors as I have included some umlauts in the example
references.

That's it? Well almost! First, you will have to make sure that the
entry type that you use corresponds to the type of document that you
are citing. Second, you will probably get some or all of your
references from standard sources such as Spires, Inspire\footnote{I
  will refer to both as Spires in this chapter} or CDS you will have
to change the entries a bit so that they get formatted the way you want.


%------------------------------------------------------------------------------
\section{\BibTeX{} entries}
\label{sec:ref:bib}

In this section I discuss how to format your \BibTeX\ databases,
i.e. the \texttt{.bib} files. In the following section I talk about
how you make your references look the way you want them to be in your
thesis.

%------------------------------------------------------------------------------
\subsection{Entry types}
\label{sec:ref:entry}

One question is what entry type you should use for what? I will
give here recommendations what to use for
\Package{biblatex}. Some of the entry types that \Package{biblatex}
has are not part of \BibTeX.

\begin{description}
\item[@article] This is easy -- use it for articles\index{article} published in
  journals, e.g.~\cite{Abramowicz:2010ih}.
\item[@book] Just as easy -- use it for books,\index{book} e.g.~\cite{kopka04}.
\item[@proceedings, @inproceedings] The name says it all. Use
  \Option{@inproceedings} for a paper in the proceedings\index{proceedings} and
  \Option{@proceedings} for the whole volume.
\item[@collection] Use it for things such as the ATLAS Technical Design
  Report~\cite{lhc:vol1}\index{report!technical} where the names that you find are the
  editors. Use \Option{@incollection} for a single article in a
  collection.
\item[@report] Use it for conference\index{conference note}\index{note!conference} and
  internal\index{internal note}\index{note!internal} notes. This is
  probably also the best type to use for preprints. You can also use
  \Option{@online}. However, then the title is usually printed in
  italics rather than upright text inside quotes.
\item[@online] Use it for things that are only available online,
  e.g.~\cite{lshort}.
\item[@thesis] The name says it all.\index{thesis}
  \Option{@phdthesis}\index{PhD thesis}\index{thesis!PhD}  and
  \Option{@mastersthesis}\index{Master thesis}\index{thesis!Master}
  also exist. If you are using \Package{biblatex} you can and should specify
  the thesis type, e.g.\ \texttt{type = \{PhD\}}, see for example a
  PhD thesis~\cite{tlodd:2012}.
\end{description}

\Package{biblatex} also knows about multivolume proceedings etc. See
the manual for more details.

Note that Spires will always give you a \BibTeX{} entry of type
\Option{@article}, so you should adjust it by hand according to what
the document you refer to really is. CDS tries a bit harder, but you
probably still have to set the entry type by hand.

As indicated above, \Package{biblatex} knows about preprint archives,
online references with a url etc.\ and can format the references so
that you can click on a DOI or arXiv number. Details of how these are
handled are well documented in the manual.  In order to make use of
these abilities you have to modify the Spires format of the references
a bit so that it is fully compatible with what \Package{biblatex}
expects for preprints etc. More details on this are given below.

What else do you have to be careful about? The first thing to know is
that \BibTeX{} will try to format your author names and titles. Thus,
if you want the title to remain in exactly the form you have typed it
in include it in \enquote{"\{Title\}"}, i.e.\ both double quotes and
braces. If not, collaborations and accelerators tend to be converted
to lowercase, e.g.\ \enquote{lhc} instead of \enquote{LHC}. If you use
an author such as ``ATLAS Collaboration'' it get printed as
\enquote{A.\ Collaboration}.


%------------------------------------------------------------------------------
\subsection{Entries from Spires and CDS}
\index{spires}\index{CDS}
\label{sec:ref:cds}

Things like the LHC Design Report\index{design report} are by default
called \Option{@article} in Spires~\cite{Bruning:2004ej-inspire} or
\Option{@book} in CDS~\cite{Bruening:782076-CDS}. They are in fact
best declared as \Option{@collection} with the \Option{author} field
replaced by \texttt{editor} and a field indicating the
\texttt{institution} instead of
\texttt{publisher}~\cite{lhc:vol1-final}. You can also add the CDS
link as a \texttt{url} field.

Conference notes,\index{conference note} e.g.\ from ATLAS, are defined
as \Option{@techreport} by CDS~\cite{ATLAS-CONF-2011-008-CDS}. It is
better to just call them \Option{@report}. You should add an author,
usually just \texttt{author = "\{ATLAS
  Collaboration\}",}~\cite{ATLAS-CONF-2011-008-final}. You may also
have to change the month format to avoid error messages. For internal
notes, also call them report and add \texttt{type = {internal report}}
to the entry. Again you could add the CDS link as a \texttt{url}
field. For preprints,\index{preprint} I also think it is best to use
the \Option{@report} entry type.

Books\index{book} need to be changed from \Option{@article} to
\Option{@book} and it is better to give the ISBN in the \texttt{isbn}
field~\cite{Halzen:1984mc-final} rather than the \texttt{reportNumber}
field as given in Spires~\cite{Halzen:1984mc-inspire}.

Theses\index{thesis} should used the \Option{@thesis} entry type and
then add a \texttt{type} field. Alternatively you can use
\Option{@mastersthesis} or \Option{@phdthesis}.

In all cases you probably have to edit the titles a bit to get
things like $\sqrt{s} = \SI{7}{\TeV}$ printed properly. An open
question is whether you should assume the use of a units package in
the formatting of the title. If you want to make your \texttt{.bib}
files usable by others, it is probably best to do the formatted by
hand.

An example of a typical ATLAS paper as it comes from
Spires~\cite{Aad:2010ey-inspire} needs a bit of work.  With \TeXLive
2011 the link to DOI and arXiv both work well~\cite{Aad:2010ey-final}.
With \TeXLive 2009, it works on my laptop, but does not work properly
on ATLAS machines (Ubuntu 10.04) that have version 0.8e of \Package{biblatex}.
It may be that one can get it to work with a bit of help, but I have
not tried all possible options.

% Depending on which\TeXLive 2009 version you actually have, you might
% have to give a bit of help for the arXiv link to
% work%~\cite{Aad:2010ey-tex2009}.


%------------------------------------------------------------------------------
\subsection{More on names}
\label{sec:ref:names}

The best way to format author names so that they appear correctly
whatever \BibTeX{} style you use is \texttt{Surname, Name}. Any other
syntax is likely to get mangled.

What about collaboration names? If you use Spires as the source of
your \BibTeX{} entries, you will see that it has a field for the
collaboration. This is often, but not always, formatted
correctly. However, very few \BibTeX{} styles pay any attention to
this field. The ones from Spires listed below will work properly. The
only other reliable alternative I have found is to use the following
syntax:
\begin{verbatim}
@Article{Chekanov:2009qja,
     author    = "{ZEUS Collab.} and Chekanov, S. and others",
\end{verbatim}
which then usually gets formatted as \enquote{ZEUS Collab., Chekanov S. et
al.,}.
I went through and changed the references in \texttt{zeus\_2009.bib}
accordingly.


%------------------------------------------------------------------------------
\section{Formatting references}
\label{sec:ref:format}

While \BibTeX{} or \Package{biblatex} format the references for you
from the \texttt{.bib} files, you have to tell them what format you
want!  For a start, you have to choose between an alphabetic and a
numeric scheme for the references. Most journals use a numeric
style. This corresponds to style \Option{unsrt} or a variant thereof
using standard \BibTeX.  If you use \Package{biblatex} you include the
package with option \Option{numeric-comp} or
\Option{numeric}. \Option{numeric-comp} produces more compact
citations (e.g.\ [1-4,7,9]) than \Option{numeric} (e.g.\
[1,2,3,4,7,9].

For this guide (for a change) I use an alphabetic style:
\Option{alpha} with \BibTeX{} or option \Option{alphabetic} with
\Package{biblatex}. In the thesis skeleton I use a more usual unsorted
numeric style.


%------------------------------------------------------------------------------
\subsection{\Package{biblatex} styles}
\label{sec:ref:bbx}

My experience with \Package{biblatex} has only recently been acquired and I
am still learning, but I have tried out a few things that I will
document here. The first official stable release was 19 Oct
2010. Active and rapid development is ongoing -- there were many
updates in 2011.  \TeXLive 2009 includes either Version 0.8e (ATLAS
cluster Ubuntu 10.04) or 1.4c (Kubuntu 11.10), so some things I
recommend below may work a bit differently depending on which version
you have. See Appendix~\ref{sec:app:tex} on how to install a newer
version of \TeXLive if you want a more up-do-date version of \LaTeX.

Looking for numeric styles you can either use the built-in
\Option{numeric} or \Option{numeric-comp}.  The \Option{numeric-comp}
style is used by default in the thesis skeleton. I made a few
adjustments that are included in the file
\texttt{./biblatex/biblatex-num-v2009.sty} or
\texttt{./biblatex/biblatex-num-v2011.sty}. Again which file is used
by default is steered by the \Macro{texlive} macro which is set in the
main file.

You can fine tune things even more by using hooks that are
available. For example, if you do not want to print the URL field you
can include the command:
\begin{verbatim}
\AtEveryBibitem{\clearfield{url}}
\end{verbatim}
in the preamble or in the relevant style file given in the previous
paragraph. It is not clear to me if you also need
\verb+\AtEveryCitekey{\clearfield{url}}+.

A fairly nice-looking style is \Option{ieee}. This is only be
available in very recent releases (2011) of \TeXLive. After playing
around a bit with the \Option{ieee} style, I decided it is too new and
has too many settings that depend on have a new version of \Package{biblatex}
for now.

There are slowly more and more \Package{biblatex} styles around, but
not as many as \BibTeX. It is, however, much easier to change things
(usually you can just change an option) with \Package{biblatex} than it
was for \BibTeX, so you can probably start with a standard file and
just make adjustments in your preamble. I have found a number of very
useful hints on how to make changes in
\url{http://tex.stackexchange.com/} -- just search for \Package{biblatex}.

If it is available, \Option{biber} is probably the preferred backend to
\Option{bibtex8}. However, the backend is mostly relevant for
sorting, so it probably does not matter which you use if you use an
option that gives the references in the order that they were
cited. \Option{biber} seems to work well with \TeXLive 2011; it is
often not available with earlier versions.

If you get an error such as:
{\scriptsize
\begin{verbatim}
biber     thesis_guide
data source /tmp/par-62726f636b/cache-ab06f20732bfab23dfa35f56998ad4edca61bee1//inc/lib/Biber/LaTeX/recode_data.xml not found in .
Compilation failed in require at Biber/Utils.pm line 21.
\end{verbatim}
}
\noindent
then you should delete the directory \texttt{/tmp/par-...} and try to
run again.

%------------------------------------------------------------------------------
\subsection{\BibTeX{} styles}
\label{sec:ref:bst}

If you use references directly from Spires, then it is probably best to
use one of the style files that is compatible with their format. A
list can be found on
\url{http://www.slac.stanford.edu/spires/hep/refs/bibstyles.shtml}. I
have used \Option{utphys} a few times and it works OK. I see that there
are also style files available there for common HEP journals, which
could save quite a bit of work. The big advantage of \Option{utphys}
is that it also knows about the arXiv and preprints.

The equivalent of \Package{biblatex}'s \Option{ieee} style in \BibTeX\ is
\Option{ieeetr}. It also knows about arXiv and preprints. However, it
does not know about collaborations.


%------------------------------------------------------------------------------
\section{Sources for references}
\label{sec:ref:sources}
%------------------------------------------------------------------------------

The ZEUS collaboration kept a reasonably up-to-date list of ZEUS and
H1 publications (as well as some others) in \BibTeX{} format. ATLAS
also keeps such a list and I assume that other collaborations keep
similar lists.

Within experimental high energy physics the standard way to get a
reference is to use Spires
(\url{http://www.slac.stanford.edu/spires/}). This is currently being
replaced by Inspire, which by now has become the standard.
One problem with Spires was that it was very slow and regularly
timed out when you perform searches. You can get the appropriate
Spires entry by using the Spires search engine. Alternatively if you
know the arXiv preprint number you can go from its entry to Spires
directly. A link to Inspire is now there instead.

To get the ZEUS references I used above I first tried the following
command in Spires:
\begin{verbatim}
find exp zeus and date 2009
\end{verbatim}
This does not really give you the references you expect though! It
seems much more reliable to use an author name so I used:
\begin{verbatim}
find a chekanov and date 2009
\end{verbatim}
and then selecting \BibTeX{} format, saving the resulting page in a file
and removing the \texttt{<pre>} and \texttt{</pre>} entries between
references. This worked better, even though I got a whole load of
ATLAS papers as well.

If you then try to use the references, you get complaints that
something is not in math mode. You have to go through by hand and
change things such as \verb+Q^2+ to \verb+$Q^{2}$+.


%------------------------------------------------------------------------------
\section{Common wishes}
\label{sec:ref:tips}
%------------------------------------------------------------------------------

It is possible that you would like to combine several articles into a
single reference. The \Package{mcite} package was designed to do this,
but is not compatible with \Package{biblatex} and
\Package{hyperref}. \Package{biblatex} has another solution that it
calls sets.

In \texttt{standard\_refs-biber.bib} and
\texttt{standard\_refs-bibtex.bib} I have put in the three standard
references for the Standard Model~\cite{gsw}. They are combined by
using \Option{@Set}\index{biblatex!Set option@"@Set option} and the relevant
keys. If you use a recent version of \texttt{biber} (\TeXLive 2011)
this is all you have to do.  If, however, you are using \TeXLive
2009, and therefore the \texttt{bibtex8} backend, the
\texttt{crossref} field must contain the same key as the first one in
\texttt{entryset}.

One things you should always do is include all references in a single
\Macro{cite}. e.g.\ there were quite a few ZEUS publications in
2009~\cite{Chekanov:2009qja,Chekanov:2009zz,Chekanov:2009tu} is better
than~\cite{Chekanov:2009qja}\cite{Chekanov:2009zz}\cite{Chekanov:2009tu}.
If you want to get a list of references printed in the form \enquote{[m--n]},
then with \Package{biblatex} you should use the style
\Option{numeric-comp}. In 2009 there were many papers published by the
ZEUS
collaboration~\cite{Chekanov:2009qja,Chekanov:2009zz,Chekanov:2009tu}
as well as several articles from both the H1 and ZEUS
collaborations\cite{Chekanov:2009wt,Aaron:2009wg}. See
Section~\ref{sec:ref:mcite} on how to do this with \BibTeX.

You are nearing the end of your thesis and have to properly format all
the references that you have. However, they are spread over several
files and these files also contain many references that you do not use
or want to correct. How best to proceed?
\begin{verbatim}
bibtool -x mythesis.aux -o refs.bib
\end{verbatim}\index{bibtool}
will extract the entries that you use and in future you can use and
correct \texttt{refs.bib}, which only contains the references that you
actually cite.\footnote{%
I got this tip from
\url{http://tex.stackexchange.com/questions/417/how-to-split-all-bibtex-referenced-entries-from-a-big-bibtex-database-to-a-copy}. Do
not forget to change \texttt{mythesis.tex} to use
\texttt{refs.bib} instead of the previous sources.}


%------------------------------------------------------------------------------
\section{Using \Package{mcite}}
\label{sec:ref:mcite}
%------------------------------------------------------------------------------

As mentioned above, the \Package{mcite} package used to be a good way
of combining several articles into a single references and also
getting them to be printed out in the form \enquote{[m--n]}, rather
than \enquote{[l,m,n]} or \enquote{[l],[m],[n]}. How do you achieve
this?  Just put all the articles in a single \Macro{cite} and prefix
those that should be lumped together with a \enquote{*}, e.g.\
\verb+\cite{Chekanov:2009wt,*Aaron:2009wg,*Aaron:2009sma}+.  The
problem is that this package does not appear to be compatible with the
\Package{hyperref} package, so you have to choose between the
two. Given the ability that the \Package{hyperref} package offers to
jump directly to sections, equations, references referred to in a
document, I guess most of you will go with \Package{hyperref} rather
than \Package{mcite}.

As mentioned above the \Package{biblatex} package offers a
more modern alternative and different ways of achieving the same
results! It is also not compatible with \Package{mcite}.

A modified version \Package{mcite} is used by ZEUS in its LaTeX4ZEUS
environment, which is why I include a short description here.

%%% Local Variables:
%%% mode: latex
%%% TeX-master: "../thesis_guide"
%%% End:

%\printbibliography[heading=subbibliography]

\include{guide_layout}
%\printbibliography[heading=subbibliography]

%-------------------------------------------------------------------------------
% Appendices etc.
\appendix

% \part*{Appendix}
% !TeX root = thesis_guide.tex
% chktex-file 1 chktex-file 46

%==============================================================================
\chapter{Changes and plans}%
\label{sec:app:changes}
%==============================================================================

\LaTeX\ file: \url{./guide_appendix.tex}\\[1ex]
\noindent
In this section I used to document briefly the major changes to this guide.
I also indicate other topics for which I would like to add some more information.
The list of changes to the style files and this guide has been moved to \File{CHANGELOG.md},
as they are quite long and this seems a more appropriate place to keep them.

Here is a list of ideas for more information that could be added to
this guide:
\begin{itemize}
\item Add drawing of Feynman graphs with \Package{axodraw} and/or
  \Package{jaxodraw}.
\item Add instructions on content of CV and summary for PhD thesis.
\end{itemize}


%==============================================================================
% TeX setup and packages
%==============================================================================

\input{guide_appendix_texsetup}


%==============================================================================
% Glossaries and acronyms
%==============================================================================

\input{guide_appendix_glossary}


%==============================================================================
% Fancy TikZ example(s)
%==============================================================================

% !TeX root = thesis_guide.tex
% chktex-file 1 chktex-file 46

%==============================================================================
\chapter{Plots with \Package{\TikZ}}%
\label{sec:app:tikz}
%==============================================================================

\LaTeX\ file: \url{./guide_appendix_tikz.tex}\\[1ex]
\noindent
\Cref{fig:tikz:syst} shows how you can produce a plot showing
the contributions of many different systematic uncertainties to a
result.

\begin{figure}[htbp]
  \centering
  \input{../tikz/zeus_Ds_systematics.tex}
  \caption[Strange $D^{**}$ systematics, fragmentation fractions]{The
    results of strange excited charm meson fragmentation fraction and
    branching ratio with systematic variations.  The individual
    systematic variations are put into groups $\delta_1-\delta_5$.}%
  \label{fig:tikz:syst}
\end{figure}




%==============================================================================
% Long tables
%==============================================================================

% !TeX root = thesis_guide.tex
% chktex-file 1 chktex-file 46

%==============================================================================
\chapter{Long tables}%
\label{sec:app:tables}
%==============================================================================

\LaTeX{} file: \url{./guide_appendix_longtable.tex}\\[1ex]
\noindent
Long and complicated tables, such as tables containing the breakdown
of the systematic error for each data point are usually put into the
appendices. Code or data cards can also be included here.
Examples of complicated typesetting have been given already in \cref{sec:table}.
In this appendix I give two examples of a table
(\cref{tab:alphabet:xtab,tab:alphabet:longtable}) that goes over more than one page
using the \Package{xtab} and \Package{longtable} packages.
Some features of the \Package{longtable} package:
\begin{itemize}\setlength{\parskip}{0pt}
  \item A \Env{longtable} is a combination of \Env{tabular} and \Env{table} in one environment. 
  \item While footnotes do not work properly in a normal \Env{tabular},
    but they do work in \Env{longtable}.
  \item You have to terminate the \Macro{caption} with \verb|\\|.
  \item If you want the caption to be at the end of the table you should include 
    it in the \Macro{endlastfoot} block.
\end{itemize}

Some features of the \Package{xtab} package:
\begin{itemize}\setlength{\parskip}{0pt}
  \item Use the \Env{mpxtabuar} environment to include footnotes in a table.
  \item You should specify the table header and footer outside the table itself.
  \item Do not include \Env{xtabular} inside a \Env{table} environment, as the table will
    then be output on one page, which is not what you want!
  \item A few things you can tweak to get the page breaks in the right place:
    According to the \Package{xtab} documentation you should first try to play
    around with the variable \Macro{xentrystretch}. The default value is \num{0.1}.
    Decrease this to put more on a page and increase it to get less.
    You can even set it to a negative value!
    The value can be set per table.
    As an alternative you can use the \Macro{shrinkheight} command.
\end{itemize}

As mentioned in \cref{sec:package}, an alternative
is the \Package{supertabular} package.

\clearpage
The relevant parts of \cref{tab:alphabet:xtab} are:
\begin{tcblisting}{listing only}
\tablefirsthead{\toprule
  \multicolumn{1}{c}{Number} &
  \multicolumn{1}{c}{Letter} &
  \multicolumn{1}{c}{Explanation}\\
  \midrule
}
\tablehead{\midrule
  \multicolumn{1}{c}{Number} &
  \multicolumn{1}{c}{Letter} &
  \multicolumn{1}{c}{Explanation}\\
  \midrule
}
\tabletail{\midrule
  \multicolumn{3}{r}{Continued on next page}\\
  \midrule
}
\tablelasttail{\bottomrule}
\bottomcaption{The alphabet set using the \Package{xtab} package.}%
\label{tab:alphabet:xtab}
\xentrystretch{-0.15}
\begin{center}
  \begin{mpxtabular}{rcl}
   1 & a & The lowercase 1st letter in the alphabet\footnote{%
    \enquote{a} deserves a footnote}\\
    ...
    26 & Z & The uppercase 26th letter in the alphabet\\
  \end{mpxtabular}
\end{center}
\end{tcblisting}

\clearpage
The relevant parts of \cref{tab:alphabet:longtable} are:
\begin{tcblisting}{listing only}
  \caption{The alphabet set using the \Package{longtable} package.%
  \label{tab:alphabet:longtable}}\\
  \begin{longtable}{rcl}
  \toprule
  \multicolumn{1}{c}{Number} &
  \multicolumn{1}{c}{Letter} &
  \multicolumn{1}{c}{Explanation}\\
  \midrule
\endfirsthead
  \midrule
  \multicolumn{1}{c}{Number} &
  \multicolumn{1}{c}{Letter} &
  \multicolumn{1}{c}{Explanation}\\
  \midrule
\endhead
  \midrule
  \multicolumn{3}{r}{Continued on next page}\\
  \midrule
\endfoot
  \bottomrule
\endlastfoot
    1 & a & The lowercase 1st letter in the alphabet\footnote{%
    \enquote{a} deserves a footnote}\\
    26 & Z & The uppercase 26th letter in the alphabet\\
\end{longtable}
\end{tcblisting}

\clearpage
\tablefirsthead{\toprule
  \multicolumn{1}{c}{Number} &
  \multicolumn{1}{c}{Letter} &
  \multicolumn{1}{c}{Explanation}\\
  \midrule
}
\tablehead{\midrule
  \multicolumn{1}{c}{Number} &
  \multicolumn{1}{c}{Letter} &
  \multicolumn{1}{c}{Explanation}\\
  \midrule
}
\tabletail{\midrule
  \multicolumn{3}{r}{Continued on next page}\\
  \midrule
}
\tablelasttail{\bottomrule}
\bottomcaption{The alphabet set using the \Package{xtab} package.}%
\label{tab:alphabet:xtab}
\xentrystretch{-0.15}
\begin{center}
  \begin{mpxtabular}{rcl}
   1 & a & The lowercase 1st letter in the alphabet\footnote{%
    \enquote{a} deserves a footnote}\\
   2 & b & The lowercase 2nd letter in the alphabet\\
   3 & c & The lowercase 3rd letter in the alphabet\\
   4 & d & The lowercase 4th letter in the alphabet\\
   5 & e & The lowercase 5th letter in the alphabet\\
   6 & f & The lowercase 6th letter in the alphabet\\
   7 & g & The lowercase 7th letter in the alphabet\\
   8 & h & The lowercase 8th letter in the alphabet\\
   9 & i & The lowercase 9th letter in the alphabet\\
  10 & j & The lowercase 10th letter in the alphabet\footnote{%
    \enquote{j} deserves another footnote}\\
  11 & k & The lowercase 11th letter in the alphabet\\
  12 & l & The lowercase 12th letter in the alphabet\\
  13 & m & The lowercase 13th letter in the alphabet\\
  14 & n & The lowercase 14th letter in the alphabet\\
  15 & o & The lowercase 15th letter in the alphabet\\
  16 & p & The lowercase 16th letter in the alphabet\\
  17 & q & The lowercase 17th letter in the alphabet\\
  18 & r & The lowercase 18th letter in the alphabet\\
  19 & s & The lowercase 19th letter in the alphabet\\
  20 & t & The lowercase 20th letter in the alphabet\\
  21 & u & The lowercase 21st letter in the alphabet\\
  22 & v & The lowercase 22nd letter in the alphabet\\
  23 & w & The lowercase 23rd letter in the alphabet\\
  24 & x & The lowercase 24th letter in the alphabet\\
  25 & y & The lowercase 25th letter in the alphabet\\
  26 & z & The lowercase 26th letter in the alphabet\\
   1 & A & The uppercase 1st letter in the alphabet\\
   2 & B & The uppercase 2nd letter in the alphabet\\
   3 & C & The uppercase 3rd letter in the alphabet\\
   4 & D & The uppercase 4th letter in the alphabet\\
   5 & E & The uppercase 5th letter in the alphabet\\
   6 & F & The uppercase 6th letter in the alphabet\\
   7 & G & The uppercase 7th letter in the alphabet\\
   8 & H & The uppercase 8th letter in the alphabet\\
   9 & I & The uppercase 9th letter in the alphabet\\
  10 & J & The uppercase 10th letter in the alphabet\\
  11 & K & The uppercase 11th letter in the alphabet\\
  12 & L & The uppercase 12th letter in the alphabet\\
  13 & M & The uppercase 13th letter in the alphabet\\
  14 & N & The uppercase 14th letter in the alphabet\\
  15 & O & The uppercase 15th letter in the alphabet\\
  16 & P & The uppercase 16th letter in the alphabet\\
  17 & Q & The uppercase 17th letter in the alphabet\\
  18 & R & The uppercase 18th letter in the alphabet\\
  19 & S & The uppercase 19th letter in the alphabet\\
  20 & T & The uppercase 20th letter in the alphabet\\
  21 & U & The uppercase 21st letter in the alphabet\\
  22 & V & The uppercase 22nd letter in the alphabet\\
  23 & W & The uppercase 23rd letter in the alphabet\\
  24 & X & The uppercase 24th letter in the alphabet\\
  25 & Y & The uppercase 25th letter in the alphabet\\
  26 & Z & The uppercase 26th letter in the alphabet\\
  \end{mpxtabular}
\end{center}

\clearpage
\begin{longtable}{rcl}
  \caption{The alphabet set using the \Package{longtable} package.
  \label{tab:alphabet:longtable}}\\
  \toprule
  \multicolumn{1}{c}{Number} &
  \multicolumn{1}{c}{Letter} &
  \multicolumn{1}{c}{Explanation}\\
  \midrule
\endfirsthead
  \midrule
  \multicolumn{1}{c}{Number} &
  \multicolumn{1}{c}{Letter} &
  \multicolumn{1}{c}{Explanation}\\
  \midrule
\endhead
  \midrule
  \multicolumn{3}{r}{Continued on next page}\\
  \midrule
\endfoot
  \bottomrule
  \caption{The alphabet set using the \Package{longtable} package.}
\endlastfoot
   1 & a & The lowercase 1st letter in the alphabet\footnote{%
    \enquote{a} deserves a footnote}\\
   2 & b & The lowercase 2nd letter in the alphabet\\
   3 & c & The lowercase 3rd letter in the alphabet\\
   4 & d & The lowercase 4th letter in the alphabet\\
   5 & e & The lowercase 5th letter in the alphabet\\
   6 & f & The lowercase 6th letter in the alphabet\\
   7 & g & The lowercase 7th letter in the alphabet\\
   8 & h & The lowercase 8th letter in the alphabet\\
   9 & i & The lowercase 9th letter in the alphabet\\
  10 & j & The lowercase 10th letter in the alphabet\footnote{%
    \enquote{j} deserves another footnote}\\
  11 & k & The lowercase 11th letter in the alphabet\\
  12 & l & The lowercase 12th letter in the alphabet\\
  13 & m & The lowercase 13th letter in the alphabet\\
  14 & n & The lowercase 14th letter in the alphabet\\
  15 & o & The lowercase 15th letter in the alphabet\\
  16 & p & The lowercase 16th letter in the alphabet\\
  17 & q & The lowercase 17th letter in the alphabet\\
  18 & r & The lowercase 18th letter in the alphabet\\
  19 & s & The lowercase 19th letter in the alphabet\\
  20 & t & The lowercase 20th letter in the alphabet\\
  21 & u & The lowercase 21st letter in the alphabet\\
  22 & v & The lowercase 22nd letter in the alphabet\\
  23 & w & The lowercase 23rd letter in the alphabet\\
  24 & x & The lowercase 24th letter in the alphabet\\
  25 & y & The lowercase 25th letter in the alphabet\\
  26 & z & The lowercase 26th letter in the alphabet\\
   1 & A & The uppercase 1st letter in the alphabet\\
   2 & B & The uppercase 2nd letter in the alphabet\\
   3 & C & The uppercase 3rd letter in the alphabet\\
   4 & D & The uppercase 4th letter in the alphabet\\
   5 & E & The uppercase 5th letter in the alphabet\\
   6 & F & The uppercase 6th letter in the alphabet\\
   7 & G & The uppercase 7th letter in the alphabet\\
   8 & H & The uppercase 8th letter in the alphabet\\
   9 & I & The uppercase 9th letter in the alphabet\\
  10 & J & The uppercase 10th letter in the alphabet\\
  11 & K & The uppercase 11th letter in the alphabet\\
  12 & L & The uppercase 12th letter in the alphabet\\
  13 & M & The uppercase 13th letter in the alphabet\\
  14 & N & The uppercase 14th letter in the alphabet\\
  15 & O & The uppercase 15th letter in the alphabet\\
  16 & P & The uppercase 16th letter in the alphabet\\
  17 & Q & The uppercase 17th letter in the alphabet\\
  18 & R & The uppercase 18th letter in the alphabet\\
  19 & S & The uppercase 19th letter in the alphabet\\
  20 & T & The uppercase 20th letter in the alphabet\\
  21 & U & The uppercase 21st letter in the alphabet\\
  22 & V & The uppercase 22nd letter in the alphabet\\
  23 & W & The uppercase 23rd letter in the alphabet\\
  24 & X & The uppercase 24th letter in the alphabet\\
  25 & Y & The uppercase 25th letter in the alphabet\\
  26 & Z & The uppercase 26th letter in the alphabet\\
\end{longtable}

% The following chapter needs some tweaks for XeLaTeX
%==============================================================================
\ifXeTeX
\chapter{A famous equation is \texorpdfstring{$\symbfit{E = mc^{2}}$}{E = mc^2}}%
\label{sec:emc2}\index{math!bold}
\else
\chapter{A famous equation is \texorpdfstring{$E = mc^{2}$}{E = mc2}}%
\label{sec:emc2}\index{math!bold}
\fi
%==============================================================================

\LaTeX{} file: \url{./guide_appendix.tex}\\[1ex]
\noindent
This chapter was included to check that one gets bold mathematics in
a chapter/section title, but not in the table of contents.

When using pdf\LaTeX, no special handling of titles should be necessary.
This is because the following line was added:
\begin{verbatim}
  \def\bfseries{\fontseries\bfdefault\selectfont\boldmath}
\end{verbatim}
%No optional title for a chapter is needed as the chapter title is typeset in bold
%font also in the table of contents.

This tweak does not work with \XeLaTeX{} and \LuaLaTeX,
so a short title may also be needed to cope with bold mathematics.

%------------------------------------------------------------------------------
\ifXeTeX
\section{A slightly less famous equation \texorpdfstring{$\symbfit{F = m a}$}{F = ma}}%
\label{sec:fma}
\else
\section{A slightly less famous equation \texorpdfstring{$F = m a$}{F = ma}}%
\label{sec:fma}
\fi
%------------------------------------------------------------------------------

The title here does not include \Macro{boldmath}, as the bold font series turns on bold math by default.
Note that the
section in the table of contents is typeset in a normal font when
writing a book or report.


%------------------------------------------------------------------------------
\ifthenelse{\boolean{XeTeX} \OR \boolean{LuaTeX}}{%
\section[Cross-section given by \texorpdfstring{$\symbf{sigma = N /}\symcal{L}}{sigma = N / L}$]%
        {The cross-section is given by \texorpdfstring{$\symbf{\sigma = N /}\symbfcal{L}$}{sigma = N / L}}%
\label{sec:sig}
}{%
\section[Cross-section given by \texorpdfstring{$\sigma = N / \mathcal{L}$}{sigma = N / L}]%
        {The cross-section is given by \texorpdfstring{$\sigma = N / \mathcal{L}$}{sigma = N / L}}%
\label{sec:sig}
}
%------------------------------------------------------------------------------

This attempt includes a Greek and a calligraphic letter to make sure they work as well.
It also includes the section title as a short form and a regular form.



%==============================================================================
% Obsolete information and instructions
%==============================================================================

\input{guide_appendix_obsolete}
%\printbibliography[heading=subbibliography]

\backmatter
%-------------------------------------------------------------------------------
% Declare bibliography, lists of figures and tables and acknowledgements as backmatter
% Chapter/section numbers are turned off
%
% Include the following lines and comment out \printbibliography if
% you use BiBTeX for the bibliography.
% If you use biblatex package the files should de specified in the preamble.
% {\raggedright
%   \bibliographystyle{unsrt}
%   \bibliography{./guide_refs.bib,../refs/standard_refs-bibtex,../refs/example_refs-latin1,%
%     ../guide/refs/zeus_2009,%
%     ../guide/refs/zeus_2010}
%  }

%-------------------------------------------------------------------------------
% Use biblatex for the bibliography
% Add bibliography to Table Of Contents
%
\printbibliography[heading=bibintoc]

\listoffigures
\listoftables

%-------------------------------------------------------------------------------
% Print the glossary
\printglossaries

%-------------------------------------------------------------------------------
% Print the index
\printindex

%-------------------------------------------------------------------------------
% Acknowledgements
\include{guide_acknowledge2}

\end{document}

%%% Local Variables:
%%% mode: latex
%%% TeX-master: t
%%% End:
